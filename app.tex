\documentclass[aps,prb,onecolumn,nofootinbib,citeautoscript,10pt]{revtex4-2}  
\synctex=1 



\usepackage{amsmath,amssymb,bm} 
\usepackage{graphicx,comment}

\usepackage[tight]{subfigure} 

\usepackage[dvipsnames]{xcolor} 
\usepackage[papersize={8.5in,11in}]{geometry}
\usepackage[colorlinks=true]{hyperref}
\hypersetup{
    bookmarks=true,         % show bookmarks bar?
    unicode=false,          % non-Latin characters 
    pdftoolbar=true,        % show Acrobat
    pdfmenubar=true,        % show Acrobat 
    pdffitwindow=false,     % window fit to page when opened
    pdfstartview={FitH},    % fits the width of the page to the window
    pdfkeywords={keyword1} {key2} {key3}, % list of keywords
    pdfnewwindow=true,      % links in new window
    colorlinks=true,       % false: boxed links; true: colored links
    linkcolor=magenta, %red,          % color of internal links (change box color with linkbordercolor)
    citecolor=blue,        % color of links to bibliography
    filecolor=magenta,      % color of file links
    urlcolor=blue           % color of external links
} 

\geometry{top=1.5cm, left= 1.5 cm, right= 1.5 cm, bottom= 1.5 cm}        



%========================= PACKAGES ==================================
\usepackage{graphicx}
\usepackage{dcolumn}
\usepackage{color}
\usepackage{amssymb,amsmath}
\usepackage{tabularx,graphicx}
\usepackage{epstopdf}
\usepackage{latexsym}
\usepackage{colortbl}
\usepackage{psfrag}
\usepackage{bbm,bm,array,physics}
\usepackage{dsfont}
\usepackage{float}




\def \nn{\nonumber \\}

\def\*#1{\mathbf{#1}} %\bold font
\newcommand{\at}[2][]{#1|_{#2}}




\begin{document}



\title{Supplementary material for ``Disentangling contributions to longitudinal magnetoconductivity for Kramers-Weyl nodes''}



\author{Ipsita Mandal}
\affiliation{Department of Physics, Shiv Nadar Institution of Eminence (SNIoE), Gautam Buddha Nagar, Uttar Pradesh 201314, India
\\ipsita.mandal@snu.edu.in}



\maketitle




 
%%%%%%%%%%%%%%%%% Apppendix %%%%%%%%%%%%%%%%%%%%%

\appendix

%%%%%%%%%%%%%%%%%%%%%%%%%%%%%%%%
\section{Berry curvature and orbital magnetic moment} 
\label{app_bc}

For the $s^{\rm th}$ band, the vector fields for the BC and the OMM are obtained using \cite{xiao_review,xiao07_valley}
\begin{align} 
& {\boldsymbol \Omega}^s ( \boldsymbol k)  = 
    i  \left[ \nabla_{ \boldsymbol k}  \psi_{s}({ \boldsymbol k}) \right ]^\dagger 
    \cross  \left [ \nabla_{ \boldsymbol k}  \psi_{s}({ \boldsymbol k}) \right ]
%%%%%%%%%%%%%%%%%%%%%%%
\text{and } 
{\boldsymbol {m}}^s ( \boldsymbol k) = 
\frac{  -\,i \, e} {2 } \,
 \left[ \nabla_{ \boldsymbol k} \psi_{s} ({ \boldsymbol k}) \right ]^\dagger 
 \cross
\Big [
\left \lbrace \mathcal{H} ({ \boldsymbol k}) -\varepsilon^s
({ \boldsymbol k}) 
\right \rbrace
\left \lbrace \boldsymbol \nabla_{ \boldsymbol k} \psi_{s}({ \boldsymbol k})
 \right \rbrace \Big  ] ,
%%%%%%%%%%%
\end{align}
respectively. For our two-band system, the expressions above simplify to \cite{fuchs-review}
%%%%%%%%%%%%%%%%%%%%%%%%%%
\begin{align} 
\label{eqomm}
\Omega^s_i ( \boldsymbol  k) = \frac{  (-1)^{s} \,  
\epsilon_{ijl}}
 {4\,| \boldsymbol  k |^3} \, \boldsymbol k  \cdot
 \left[   \partial_{k_j} \boldsymbol k \cross  
 \partial_{k_l } \boldsymbol k \right ] \text{ and }
%%%%%%%%%%%%%%%%%%%%%%%
\boldsymbol m^s ( \boldsymbol  k) = 
 e\left( s \, v_0 \, k\right) \boldsymbol \Omega \,.
\end{align}
%%%%%%%%%%%%%%%%%%%%%%%%%%%%%%
% pg-52, Fuchs thesis
On evaluating the expressions in Eq.~\eqref{eqomm} for $\mathcal{H} ( \boldsymbol  k) $, we get
\begin{align}
\label{eqbcomm}
{\boldsymbol \Omega}^s( \boldsymbol k) = 
\frac{ -\, s  } { 2\, k^3 }
 \left \lbrace k_x, \, k_y, \,  k_z \right \rbrace, \quad 
%%%%%%%%%%
{\boldsymbol m}^s( \boldsymbol k) = 
\frac{ -\,  e\, v_0 } {2 \, k^2 } 
 \left \lbrace k_x, \, k_y, \,  k_z \right \rbrace .
\end{align}



%%%%%%%%%%%%%%%%%%%%%%%%
\section{Kinetic equations driven by electromagnetic fields}


The kinetic equations are governed by the Hamilton's equations of motion for the electronic quasiparticles, occupying a given Bloch band. They are described by \cite{mermin, sundaram99_wavepacket, li2023_planar, ips-kush-review, ips-rsw-ph, ips-shreya}
%%%%%%%%%%%%%%%%%%%
\begin{align}
\label{eqrkdot}
\dot {\boldsymbol r} & = \nabla_{\boldsymbol k} \, \xi_s  
- \dot{\boldsymbol k} \, \cross \, \boldsymbol \Omega^s  (\boldsymbol k)
 \text{ and }  
\dot{\boldsymbol k} = -\, e  \left( {\boldsymbol E}  
+ \dot{\boldsymbol r} \, \cross\, {\boldsymbol B} 
	\right ) \nn
%%%%%%%%%%%%%%%%%%%%%%%%%%%
\Rightarrow & \, \dot{\boldsymbol r}  = \mathcal{D}_s ({\boldsymbol k})
	\left[   \boldsymbol{w}_s ({\boldsymbol k}) +
	 e \, {\boldsymbol E}  \cross  
\boldsymbol \Omega^s ({\boldsymbol k})   + 
	 e   \left \lbrace \boldsymbol \Omega^s (\boldsymbol k )\cdot 
 \boldsymbol{w}_s ({\boldsymbol k}) 
 \right \rbrace  \boldsymbol B  \right] 
%%%%%%%%%%%%%%%%%%%%%%%%%%	  
 \text{ and } 
\dot{\boldsymbol k}  = -\, e \,\mathcal{D}_s ({\boldsymbol k})
  \left[   {\boldsymbol E} 
+   \boldsymbol{w}_s ({\boldsymbol k}) \cross  {\boldsymbol B} 
+  e  \left (  {\boldsymbol E}\cdot  {\boldsymbol B} \right )  
\boldsymbol \Omega^s  ({\boldsymbol k})\right],
\end{align}
where
\begin{align}
\boldsymbol{w}_s ({\boldsymbol k}) = {\boldsymbol  v}^s(\boldsymbol k )
+  {\boldsymbol   v}_{(m)} ({\boldsymbol k} ) \,.
\end{align}
%%%%%%%%%%%%%%%%%%%
The presence of $ {\boldsymbol \Omega}^s $ and $ \mathcal{D}_s $ reflect the nontrivial role played by a nonzero BC, as compared to the scenarios when the BC vanishes (see, for example, the systems discussed in Refs.~\cite{ips-kush, ips_tilted_dirac}). In particular, an extra term in the form of $- \dot{\boldsymbol k} \, \cross \, \boldsymbol \Omega^s $ represents an anomalous velocity, with the BC playing the counterpart of the magnetic field, albeit in the momentum space.  


%%%%%%%%%%%%%%%%%%%%%%%%
\section{Results obtained from relaxation-time approximation}
\label{secsig}

In this section, we demonstrate the analytical results for the conductivity, obtained from the simplistic application of the relaxation-time approximation. We have used this formalism in our earlier works \cite{ips_rahul_ph_strain, rahul-jpcm, ips-ruiz, ips-rsw-ph, ips-shreya, ips-tilted, ips-internode, ips-spin1-ph} to derive the in-plane longitudinal and planar-Hall components. In order to obtain closed-form analytical expressions, we have to expand the $B $-dependent terms upto a given order in $B$, assuming it has a small magnitude, which is anyway required to justify ignoring Landau-level splitting. With this in mind, the following two quantities are expanded as shown below:
%%%%%%%%%%%%%%%%%%%%%%%%%%%
\begin{align}
& f_0^\prime \left(\xi_s\right) = 
f_0^\prime \big(  {\tilde \varepsilon}_s ({ \boldsymbol k}) \big)
+ \varepsilon^{(m)} \, f_0^{\prime \prime } \big ( {\tilde \varepsilon}_s ({ \boldsymbol k})\big )
+ \frac{1}{2} \left[ \varepsilon^{(m)} \right ]^2 
f_0^{\prime \prime \prime}\big ( {\tilde \varepsilon}_s  ({ \boldsymbol k})\big )
 + \mathcal{O} (B^3 )\,, 
%%%%%%% 
\text{ where }
 {\tilde \varepsilon}_s  ({ \boldsymbol k})=
 \varepsilon^s  ({ \boldsymbol k}) +  \frac{ e \,g\,\mu_B \,s \, B} {2}\,,
 %%%%%%%%%%%%%%%%%
\nn & \text{ and } \mathcal{D}_s = \sum \limits_{n=0}^{2} 
\left [ -e \,  \Omega^z_s \, B \right ]^n 
+ \mathcal{O} (B^3 )\,.
\end{align}
%%%%%%%%%%%%%%%%%%%%
%%%%%%%%%%%%%%%%%%%
Here, the ``prime'' symbol denotes the operation of partial-differentiation, with respect to the variable shown explicitly within the brackets [e.g., $ f_0^\prime (\varepsilon) \equiv \partial_\varepsilon f_0 (\varepsilon)$]. 
Since we are working in the $T \rightarrow 0 $ limit, we have to use $f_0^\prime (\varepsilon) \rightarrow -\, 
\delta ( \varepsilon - \mu )$. Observing that the radial part of the integrals is with respect to the variable $k$, we need to use the following forms of the concerned expressions:
\begin{align}
& f_0^{\prime \prime } \big ( {\tilde \varepsilon}_s  ({ \boldsymbol k})\big )
= \frac{ \partial_k f_0^\prime \big ( {\tilde \varepsilon}_s  ({ \boldsymbol k})\big ) }
{2 \, c \, k  + s \, v_0 } \,, \quad
%%%%%%%%%%%%%%%%%%%
f_0^{\prime \prime \prime} \big ( {\tilde \varepsilon}_s  ({ \boldsymbol k})\big )
= \frac{ \partial^2_k f_0^\prime \big ( {\tilde \varepsilon}_s  ({ \boldsymbol k})\big )
}  {v_0^3 \, (2 \, c\, k + s\,  v_0 )^2 }
-
\frac{2 \,c\, \partial_k f_0^\prime \big ( {\tilde \varepsilon}_s  ({ \boldsymbol k})\big )
}  { v_0^3 \, (2 \, c\, k + s\,v_0 )^3} \,.
\end{align}
%%%%%%%%%%%%%%%%%%%%%
We refer to Refs.~\cite{ips-kush-review, ips_rahul_ph_strain, ips-rsw-ph} for a review of the generic expressions for the in-plane (i.e., coplanar with the applied electromagnetic fields) components the conductivity tensor.
Noting that the in-plane anomalous-Hall part vanishes \cite{ips-ruiz, ips-rsw-ph, ips-spin1-ph}, the generic expression therein can be expressed as
%%%%%%%%%%%%%%%%
\begin{align}
\label{eqsig}
   \sigma^s_{i j}
&= - \,e^2 \, \tau  
\int \frac{ d^3 \boldsymbol k}{(2\, \pi)^3 } \, \mathcal{D}_s
\left[  (w_s)_i \, + (W_s)_i \right ]
\left [ (w_s)_j \, + (W_s)_j \right] \,
f^\prime_0 \big (\xi_s \big )  \,,
\text{ where }
%%%%%%%%%%%%%%%%%%%%%%%
\boldsymbol{W}_s 
= e \left  ( {\boldsymbol w}_s \cdot 
  \boldsymbol {\Omega}^s \right  ) \boldsymbol{B}\,.
\end{align}
%%%%%%%%%%%%%%%%%%%



For our KWN system, only the longitudinal component survives, leading to
%%%%%%%%%%%%%%%%%%%%%%%
\begin{align}
\label{eqrelax}
 & \sigma^s_{zz} = \frac{e^2 \, \tau } {8 \,\pi ^3} 
 \left (
\sigma_{zz}^{0,s} + \sigma_{zz}^{{\rm bc},s}
+ \sigma_{zz}^{{\rm m},s} \right) ,
%%%%%%%%%%%%%%%%%%%%
\nn & \sigma_{zz}^{0,s}  =
\frac{v_0^2 \left(\sqrt{v_0}
-\sqrt{\frac{4 \,c \,{\tilde \mu} }{v_0}+v_0}\right)^2
 \left[\sqrt{\frac{4\, c\, {\tilde \mu} } {v_0}+v_0}
 +(s-1) \,  \sqrt{v_0} \right]^2}
 {6 \,c^2 \,\sqrt{4 \,c\, {\tilde \mu} +v_0^2}}\,,
%%%%%%%%%%%%%%%%%%%%%%%%%%%%
\nn & \sigma_{zz}^{{\rm bc},s}  =
\frac{2 \, e^2 \,c^2 \, B^2 }  
 {15 \,v_0 \,\sqrt{4 \,c\, {\tilde \mu} + v_0^2}}
\frac{  
2 \,(80\, s-77)\, v_0^{\frac{3}{2}}\,
\sqrt{\frac{4 \,c \,{\tilde \mu} } {v_0} + v_0}
+ 4 \,c \,{\tilde \mu}  \, (63-20 \, s) + (177-160 \,s)\, v_0^2 }
   { \left(\sqrt{v_0}-\sqrt{\frac{4 \, c  \,{\tilde \mu} }   { v_0} + v_0}
 \right)^2}\,,
 %%%%%%%%%%%%%%%%%%%%%%%%%%%
\nn & \sigma_{zz}^{{\rm m},s}  =
\frac{2  \,e^2 \, c^2 }
{15 \,\sqrt{v_0} \;\sqrt{\frac{4 \,c \,{\tilde \mu} }{v_0} + v_0 } 
\left(4 \,c \,{\tilde \mu} +v_0^2\right)^{5/2}
   \left(\sqrt{v_0}-\sqrt{\frac{4 \,c \,{\tilde \mu} }{v_0}+v_0}\right)^2}
  %%%%%%%%%
\nn & \hspace{ 1.2 cm } \times   \Bigg [
 128 \,c^3 \,{\tilde \mu}^3 \,(9 \,s-10) + 16 \,c^2\, {\tilde \mu} ^2\, (136\, s-133) \, v_0^2 
+ 216\, c^2 \,{\tilde \mu} ^2\, (5-6\, s)\, v_0^{\frac{3}{2}}
 \,\sqrt{\frac{4 \,c \,{\tilde \mu} }{v_0} + v_0}
  + (118\, s-111) \,v_0^6
 \nn & \hspace{1.75 cm}
 %%%%%%%%%%
 + 2\, (61-70 \,s ) \,v_0^{\frac{11}{2}}
 \, \sqrt{\frac{4 \,c \,{\tilde \mu} }{v_0}+v_0}  
 + 16 \,c \,{\tilde \mu}\,  (59 \,s-56) \,v_0^4
% \nn & \hspace{2 cm}
 %%%%%%%%%%
 +2\, c\, {\tilde \mu} \, (361-424 \,s)\,
   v_0^{\frac{7}{2}}\, \sqrt{\frac{4 \,c \,{\tilde \mu} }{v_0}+v_0} \,
 \Bigg ]  \,,
 %%%%%%%%%%%%
\nn & \text{where } \tilde \mu = \mu - \frac{ e \,g\,\mu_B \,s \, B} {2} \,.
\end{align}
Here, the superscipts ``bc'' and ``m'' stand for the BC-only and OMM-sourced contributions, respectively. We note that $ \sigma_{zz}^{{\rm dr},s} \equiv \sigma_{zz}^{{\rm 0},s} \vert_{B=0}$ represents the Drude part (i.e., the residual conductivity in the absence of a magnetic field).




%%%%%%%%%%%%%%%%%%%%%%%%%%%%%%%
\bibliography{ref_ksm.bib}




\end{document}
